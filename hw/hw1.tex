\documentclass[12pt]{article}
% -------------------
% Packages
% -------------------
\usepackage{
	amsmath,			% Math Environments
	amssymb,			% Extended Symbols
	amsthm,			    % Theorem Environments
	cancel,			    % Use Cancels
	enumerate,		    % Enumerate Environments
	graphicx,			% Include Images
	lastpage,			% Reference Lastpage
	multicol,			% Use Multi-columns
	multirow,			% Use Multi-rows
	xcolor,			    % Use Colors
	float,
	geometry
}

% -------------------
% Header & Footer
% -------------------
\usepackage{fancyhdr}

\newcommand{\assignment}[1]{
	\fancypagestyle{title}{
		%Headers
		\fancyhead[L]{Jake Bridge}
		\fancyhead[C]{MATH 418}
		\fancyhead[R]{HW #1}
		\renewcommand{\headrulewidth}{0.2pt}
		%Footers
		\fancyfoot[L]{}
		\fancyfoot[C]{}
		\fancyfoot[R]{}
		\renewcommand{\footrulewidth}{0.0pt}
	}
	\thispagestyle{title}
}
\fancypagestyle{pages}{
	%Headers
	\fancyhead[L]{}
	\fancyhead[C]{}
	\fancyhead[R]{}
	\renewcommand{\headrulewidth}{0.0pt}
	%Footers
	\fancyfoot[L]{}
	\fancyfoot[C]{}
	\fancyfoot[R]{}
	\renewcommand{\footrulewidth}{0.0pt}
}
\headheight=18pt
\footskip=14pt

\pagestyle{pages}

% ---------------
% Page Layout
% ---------------
\geometry{letterpaper,
bindingoffset=0.2in,
left=1in,
right=1in,
top=1in,
bottom=1in,
footskip=.25in}
% -------------------
% Font
% -------------------
%\usepackage[T1]{fontenc}
%\usepackage{charter}

%\usepackage[T1]{fontenc}
%\usepackage{mathpazo}

%\usepackage[bitstream-charter]{mathdesign}
%\usepackage[T1]{fontenc}


% -------------------
% Commands
% -------------------

% Problem Labels
\newcounter{problem}[section]
\newenvironment{problem}[1][\theproblem]{\refstepcounter{problem}\noindent \textbf{Problem~#1.\quad}}{\vskip\smallskipamount}


\newenvironment{solution}{\noindent \textbf{Solution.\quad}}{\vskip\bigskipamount}


% Special Characters
\newcommand{\N}{\mathbb{N}}
\newcommand{\Z}{\mathbb{Z}}
\newcommand{\Q}{\mathbb{Q}}
\newcommand{\R}{\mathbb{R}}
\newcommand{\C}{\mathbb{C}}

\newcommand{\mc}[1]{\mathcal{#1}}

\newcommand{\ob}[1]{\text{ob}(#1)}

% Math Operators

% Special Commands



% DOC
\begin{document}
	\assignment{1}

\begin{problem}[12]

	\begin{enumerate}
		\item 	Prove if $\mc{S}_1 \subseteq \mc{C}_1$ and $\mc{S}_2 \subseteq \mc{C}_2$ are subcategories, that $\mc{S}_1 \times \mc{S}_2$ is a subcategory of $\mc{C}_1 \times \mc{C}_2$. 
		\item Does the same hold for disjoint unions?
	\end{enumerate}
\end{problem}

\begin{solution}

We inherit a notion of composition and identity from the ambient categories, so we really just need to check that $\mc{S}_1 \times \mc{S}_2$ is closed under composition and has identities. 

There's a good bit of defining to do...Let $A_1, B_1 \in \mc{S}_1, A_2, B_2 \in \mc{S}_2$ with corresponding identities $1_{A_1}, 1_{A_2}$, etc. and morphisms $f, f' : A_1\to B_1, g, g' : A_2 \to B_2$.  

Let's take care of the identities first. 
I claim that $1_A = (1_{A_1}, 1_{A_2})$ is an identity for $A = (A_1, A_2)$. and similarly for $B = (B_1, B_2)$. Everything mentioned so far is in $\mc{S}_1 \times \mc{S}_2$ by construction and definition of subcategory. Consider any morphism $h = (f, g)$. Then, $h \circ 1_A = (f \circ 1_{A_1}, g \circ 1_{A_2}) = (f, g)$. The same occurs on the other side with the same logic. 

Now, we ought to check composition. Within $\mc{S}_1 \times \mc{S}_2$, let $h: A\to B$ and $h' : B\to C$ in the product category. $h' \circ h = (f', g') \circ (f, g) = (f' \circ f, g' \circ g)$, which is in $\mc{S}_1 \times \mc{S}_2$. 

Hence, $\mc{S}_1 \times \mc{S}_2$ is a subcategory of $\mc{C}_1 \times \mc{C}_2$. \qed

I also claim that the same holds for disjoint unions, as any subcategory of a single one of the two categories is pretty clearly a subcategory of the disjoint union of the ambient. Given two subcategories, their disjoint union ought to be a subcategory of the disjoint union of the ambient categories. The toughest part ends up being notation, so please pardon the following abomination.

In the disjoint union case, we inherit our objects and morphisms, so composition and identities are well defined. I claim that the identity within $\mc{S}_1 \coprod \mc{S}_2 (A, A)$ is the identity from whichever $\mc{S} A$ was from. WLOG $A$ is in $\mc{S}_1$. Denote $1_A \in \mc{S}_1$ by $1'_A$ and $f : A\to B$, where $B \in \mc{S}_1$. "Tagging" the elements I'll write $(1, f)$ to denote $f$ from $\mc{S}_1$, and similarly for other items. Then, $(1, f) \circ (1, 1'_A = (1, f)$. Then, $1_A \in \mc{S}_1 \coprod \mc{S}_2(A,A) = (1, f)$. Hence, we are good with identities. 

The only time composition of morphisms is sensible is when both objects are from the same category initially. Again, WLOG, suppose we have $A, B, C \in \mc{S}_1$ with morphisms $f : A \to B, g: b \to C$. Then, $(1, g) \circ (1, f) = (1, g \circ f)$, which is in the disjoint union because, again, $\mc{S}_1$ is a subcategory. Hence, composition is taken care of!

Thus, $\mc{S}_1 \coprod \mc{S}_2$ is a subcategory of $\mc{C}_1 \coprod \mc{C}_2$. \qed


\end{solution}





\begin{problem}[8]
	Given a set $X$, let $\mc{P}X$ be the power set of $X$. Build a category $\Gamma$ with the following properties:
	\begin{enumerate}
		\item $\ob{\Gamma}$ consists of all finite sets
		\item A morphism $S \rightsquigarrow T$ is a function $f: S \to \mc{P}T$ such that for any $s_1 \neq s_2 \in S, f(s_1) \cap f(s_2) = \emptyset$. 
	\end{enumerate}
	
\end{problem}

\begin{solution}
	The objects and morphisms of our category have already been defined. So, I  need to define composition and identities and check the axioms. 
	
	Given $R \xrightarrow{g} S \xrightarrow{f} T$, I define a function $f \circ g: R \to \mc{P}T$ such that $\forall r \in R, (f \circ g)(r) := \bigcup_{s \in g(r)} f(s)$. We need to check that such a morphism satisfies our rule above. Suppose $r_1 \neq r_2 \in R$. Let $t_1 \in (f \circ g)(r_1), t_2 \in (f \circ g)(r_2)$. If we can show that $t_1$ cannot equal $t_2$, that would mean the two sets they come from are disjoint. By our construction, $t_1$ must have come from one of the sets $f(s_1)$ for some $s_1 \in g(r_1)$. Similarly, $t_2$ comes from some $f(s_2)$ for some $s_2 \in g(r_2)$. But, by definition of $g$, $g(r_2) \cap g(r_1) = \emptyset$, and so $s_1 \neq s_2$. Similarly, by definition of $f$, we conclude that $f(s_1) \cap f(s_2) = \emptyset$, and so $t_1 \neq t_2$. Hence, $(f \circ g)(r_1) \cap (f \circ g)(r_2) = \emptyset$. 
	
	With our composition rule established, we'll show associativity first.
	
	Given $Q \xrightarrow{h} R \xrightarrow{g} S \xrightarrow{f} T$
	
	$(f \circ (g \circ h))(q) = \bigcup_{s \in (g \circ h)(q)} f(s) = \bigcup_{s \in \bigcup_{r \in h(q)}g(r)} f(s)  = \bigcup_{r\in h(q)}\bigcup_{s \in g(r)} f(s) = \bigcup_{r\in h(q)} (f \circ g)(r) = ((f \circ g) \circ h)(q)$, as desired!
	
	As identities, we will take the function $1_S: S \to \mc{P}S$ where $s \mapsto \{s\}$. Indeed, given $S \xrightarrow{f} T$, $(1_T \circ f)(s) = \bigcup_{t \in f(s)} 1_T(t) = f(s)$ and $(f \circ 1_S)(s) = \bigcup_{s' \in 1_S(s')} f(s') = f(s)$. Hence, our identities are indeed identities. Thus, $\Gamma$ is a category! Yay! \qed
	
\end{solution}

\begin{problem}[5]
gwuh
\end{problem}


\end{document}
