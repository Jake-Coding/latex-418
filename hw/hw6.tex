\documentclass[12pt]{article}
% -------------------
% Packages
% -------------------
\usepackage{
	amsmath,			% Math Environments
	amssymb,			% Extended Symbols
	amsthm,			    % Theorem Environments
	cancel,			    % Use Cancels
	enumerate,		    % Enumerate Environments
	graphicx,			% Include Images
	lastpage,			% Reference Lastpage
	multicol,			% Use Multi-columns
	multirow,			% Use Multi-rows
	xcolor,			    % Use Colors
	float,
	geometry,
	quiver
	}

% -------------------
% Header & Footer
% -------------------
\usepackage{fancyhdr}

\newcommand{\assignment}[1]{
	\fancypagestyle{title}{
		%Headers
		\fancyhead[L]{Jake Bridge}
		\fancyhead[C]{MATH 418}
		\fancyhead[R]{HW #1}
		\renewcommand{\headrulewidth}{0.2pt}
		%Footers
		\fancyfoot[L]{}
		\fancyfoot[C]{}
		\fancyfoot[R]{}
		\renewcommand{\footrulewidth}{0.0pt}
	}
	\thispagestyle{title}
}
\fancypagestyle{pages}{
	%Headers
	\fancyhead[L]{}
	\fancyhead[C]{}
	\fancyhead[R]{}
	\renewcommand{\headrulewidth}{0.0pt}
	%Footers
	\fancyfoot[L]{}
	\fancyfoot[C]{}
	\fancyfoot[R]{}
	\renewcommand{\footrulewidth}{0.0pt}
}
\headheight=18pt
\footskip=14pt

\pagestyle{pages}

% ---------------
% Page Layout
% ---------------
\geometry{letterpaper,
bindingoffset=0.2in,
left=1in,
right=1in,
top=1in,
bottom=1in,
footskip=.25in}
% -------------------
% Font
% -------------------
%\usepackage[T1]{fontenc}
%\usepackage{charter}

%\usepackage[T1]{fontenc}
%\usepackage{mathpazo}

%\usepackage[bitstream-charter]{mathdesign}
%\usepackage[T1]{fontenc}


% -------------------
% Commands
% -------------------

% Problem Labels
\newcounter{problem}[section]
\newenvironment{problem}[1][\theproblem]{\refstepcounter{problem}\noindent \textbf{Problem~#1.\quad}}{\vskip\smallskipamount}


\newenvironment{solution}{\noindent \textbf{Solution.\quad}}{\vskip\bigskipamount}


% Special Characters
\newcommand{\N}{\mathbb{N}}
\newcommand{\Z}{\mathbb{Z}}
\newcommand{\Q}{\mathbb{Q}}
\newcommand{\R}{\mathbb{R}}
\newcommand{\C}{\mathbb{C}}

\newcommand{\mc}[1]{\mathcal{#1}}

\newcommand{\ob}[1]{\text{ob}(#1)}
\newcommand{\natt}{\Rightarrow}
\newcommand{\iso}{\cong}
\newcommand{\op}{\text{op}}

\newcommand{\Set}{\textbf{Set}}
\newcommand{\Hom}{\textbf{Hom}}

\newcommand{\eps}{\varepsilon}


% Math Operators

% Special Commands



% DOC
\begin{document}
	\assignment{0}

\begin{problem}[55]
	Let $S, T$ be posets (as categories). What is an adjunction in these categories?
\end{problem}
\begin{solution}
	
	Suppose $F: S \to T$ is a functor.  Then, $U: T\to S$ is its right adjoint if and only if
	$\Hom_T(Fs, t) \iso \Hom_S(s, Ut)$ for all $s \in S, t \in T$.
	
	We can think of $F$ as an order preserving function. That is, $\Hom_T(Fs, t)$ is equal to $\{*_{Fs, t}\}$ if $Fs \leq_T t$ and empty otherwise. Similarly, $\Hom_S(s, Ut)$ is $*_{s, Ut}$ iff $s \leq_S Ut$. 
	
	So, $\Hom_T(Fs, t) \iso \Hom_S(s, Ut)$ for all $s \in S, t \in T$ if and only if $\forall s, t, Fs \leq_T t \iff s \leq_S Ut$. 
	
	For the sake of completeness, we will check that this (trivial) isomorphism is natural. (I will only check naturality in $s$. $t$ is almost identical.)Let the name of the isomorphism be $\alpha$ that sends $*_{Fs, t} \mapsto *_{s, Ut}$. Let $f: 's\to s$ be the statement $s' \leq s$. Then, by transitivity of the poset operation, we conclude that the following commutes (we trace the only possible morphism around the square)
	
% https://q.uiver.app/#q=WzAsOCxbMCwwLCJcXEhvbV9UKEZzLCB0KSJdLFszLDAsIlxcSG9tX1QoRnMnLCB0KSJdLFszLDMsIlxcSG9tX1MocycsIFV0KSJdLFswLDMsIlxcSG9tX1MocywgVXQpIl0sWzEsMSwiKl97RnMsIHR9Il0sWzIsMSwiKl97RnMnLCB0fSJdLFsyLDIsIipfe3MnLCBVdH0iXSxbMSwyLCIqX3tzLCBVdH0iXSxbMCwxLCItXFxjaXJjIEZmIl0sWzEsMiwiXFxhbHBoYV97cyd9Il0sWzAsMywiXFxhbHBoYV9zIiwyXSxbMywyLCItIFxcY2lyYyBmIiwyXSxbNCw1XSxbNSw2XSxbNCw3XSxbNyw2XV0=
\[\begin{tikzcd}[cramped]
	{\Hom_T(Fs, t)} &&& {\Hom_T(Fs', t)} \\
	& {*_{Fs, t}} & {*_{Fs', t}} \\
	& {*_{s, Ut}} & {*_{s', Ut}} \\
	{\Hom_S(s, Ut)} &&& {\Hom_S(s', Ut)}
	\arrow["{-\circ Ff}", from=1-1, to=1-4]
	\arrow["{\alpha_s}"', from=1-1, to=4-1]
	\arrow["{\alpha_{s'}}", from=1-4, to=4-4]
	\arrow[from=2-2, to=2-3]
	\arrow[from=2-2, to=3-2]
	\arrow[from=2-3, to=3-3]
	\arrow[from=3-2, to=3-3]
	\arrow["{- \circ f}"', from=4-1, to=4-4]
\end{tikzcd}\]

This is a bit overkill, but confirms that $F \dashv U$. \qed

\end{solution}

	\newcommand{\Mon}{\textbf{Mon}}
\begin{problem}[58]

	Show that the free functor from $\Set$ to $\Mon$ is left adjoint to the underlying set (forgetful) functor from $\Mon$ to $\Set$. 
	
	Personal note: do it with hom-set and unit-counit.
\end{problem}
\begin{solution}
	We discussed the functor $F: \Set \to \Mon$ in class and gave its action on monoids. On set functions $f : A \to B$, we define $Ff : FA \to FB$ by $Ff((n; x_1, x_2, \dots, x_n)) = (n; f(x_1), f(x_2), \dots f(x_n))$. This is clearly a monoid homomorphism in $FB$, as $Ff((n; x) \circ (m, y)) = (n+m; f(x), f(y)) = (n; f(x)) \circ (m; f(y))$. It also clearly respects identities and function composition in $\Set$. 
	
	We also discussed the function $U: \Mon \to \Set$ and gave its action on monoids. On monoid homomorphisms, $U(\phi : F \to G)$ is just $\phi$ on the underlying set. Every monoid homomorphism is just a set function with some fancy rules. 
	
	We'll do the unit-counit proof first. We first need a unit $\eta : 1_\Set \natt UF$ and counit $\eps: FU \natt 1_\Mon$. $\eta$ will have components for every set $\eta_A : A \to UFA$. We define $\eta_A(x) = (1; x)$. We also define $\eps_M: FUM \to M$ with $\eps((n; x_1, x_2, \dots x_n)) = x_1x_2\dots x_n$ where multiplication in the RHS is in $M$. 
	
	We'll take it one diagram at a time.
	% https://q.uiver.app/#q=WzAsMyxbMCwwLCJGIl0sWzIsMCwiRlVGIl0sWzIsMiwiRiJdLFswLDEsIkZcXGV0YSIsMCx7ImxldmVsIjoyfV0sWzEsMiwiXFxlcHMgRiIsMCx7ImxldmVsIjoyfV0sWzAsMiwiMV97W1xcU2V0LCBcXE1vbl19IiwyLHsibGV2ZWwiOjJ9XV0=
	\[\begin{tikzcd}
		F && FUF \\
		\\
		&& F
		\arrow["{F\eta}", rightarrow, from=1-1, to=1-3]
		\arrow["{1_{[\Set, \Mon]}}"', rightarrow, from=1-1, to=3-3]
		\arrow["{\eps F}", rightarrow, from=1-3, to=3-3]
	\end{tikzcd}\]
	 For this to commute, we need that the following commutes: 
	 % https://q.uiver.app/#q=WzAsMyxbMCwwLCJGQSJdLFsyLDAsIkZVRkEiXSxbMiwyLCJGQSJdLFswLDEsIntGXFxldGF9QSA6PSBGKFxcZXRhX0EpIl0sWzAsMiwiezFfXFxNb259IiwyXSxbMSwyLCJ7XFxlcHMgRn1BID0gXFxlcHNfe0ZBfSJdXQ==
	 \[\begin{tikzcd}
	 	FA && FUFA \\
	 	\\
	 	&& FA
	 	\arrow["{{F\eta}A := F(\eta_A)}", from=1-1, to=1-3]
	 	\arrow["{{1_\Mon}}"', from=1-1, to=3-3]
	 	\arrow["{{\eps F}A = \eps_{FA}}", from=1-3, to=3-3]
	 \end{tikzcd}\]
	 
	 Let $x \in A$
	 
	 $\eps_{FA}\circ F(\eta_A)(Fx) = \eps_{FA}((1; x)) = x$. So, the diagram commutes. 
	
	
	
%	
%		\[\begin{tikzcd}
%		U && UFU \\
%		\\
%		&& U
%		\arrow["{\eta U}", rightarrow, from=1-1, to=1-3]
%		\arrow["{1_{[\Mon, \Set]}}"', rightarrow, from=1-1, to=3-3]
%		\arrow["{U\eps}", rightarrow, from=1-3, to=3-3]
%	\end{tikzcd}\]
	
	On an individual element, we get
	

\end{solution}

\begin{problem}[60]
	Is there a free-functor-shaped left adjoint to the forgetful functor from $\textbf{Field}_p$ to $\Set$ for a fixed p?
\end{problem}
\begin{solution}
	
\end{solution}


\begin{problem}[62]
	If $F \dashv U$, show that the counit of the adjunction is invertible iff $U$ is fully faithful. Prove a dual statement about the unit.
\end{problem}
\begin{solution}
	
\end{solution}


\begin{problem}[65]
	If $F \dashv U$ and $F' \dashv U'$, then $F'F \dashv UU'$.
\end{problem}
\begin{solution}
	
\end{solution}



\begin{problem}[67]
	Construct a 2-category \textbf{Adj} with objects small categories, 1-cells from $\mc{A}\to\mc{B}$ given by pairs of adjunct functors $F: \mc{A}\to\mc{B}, U: \mc{B}\to\mc{A}$ with $F \dashv U$ , and 2-cells $\alpha: F_1 \dashv U_1 \natt F_2 \dashv U_2$ given by $\alpha: F_1 \natt F_2$. 
\end{problem}
\begin{solution}
	
\end{solution}


\begin{problem}[68]
	Show the functor $(-)^\op$ is self-adjoint.
\end{problem}
\begin{solution}
	
\end{solution}



\end{document}
