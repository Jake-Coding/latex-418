\documentclass[12pt]{article}
% -------------------
% Packages
% -------------------
\usepackage{
	amsmath,			% Math Environments
	amssymb,			% Extended Symbols
	amsthm,			    % Theorem Environments
	cancel,			    % Use Cancels
	enumerate,		    % Enumerate Environments
	graphicx,			% Include Images
	lastpage,			% Reference Lastpage
	multicol,			% Use Multi-columns
	multirow,			% Use Multi-rows
	xcolor,			    % Use Colors
	float,
	geometry,
	quiver
	}

% -------------------
% Header & Footer
% -------------------
\usepackage{fancyhdr}

\newcommand{\assignment}[1]{
	\fancypagestyle{title}{
		%Headers
		\fancyhead[L]{Jake Bridge}
		\fancyhead[C]{MATH 418}
		\fancyhead[R]{HW #1}
		\renewcommand{\headrulewidth}{0.2pt}
		%Footers
		\fancyfoot[L]{}
		\fancyfoot[C]{}
		\fancyfoot[R]{}
		\renewcommand{\footrulewidth}{0.0pt}
	}
	\thispagestyle{title}
}
\fancypagestyle{pages}{
	%Headers
	\fancyhead[L]{}
	\fancyhead[C]{}
	\fancyhead[R]{}
	\renewcommand{\headrulewidth}{0.0pt}
	%Footers
	\fancyfoot[L]{}
	\fancyfoot[C]{}
	\fancyfoot[R]{}
	\renewcommand{\footrulewidth}{0.0pt}
}
\headheight=18pt
\footskip=14pt

\pagestyle{pages}

% ---------------
% Page Layout
% ---------------
\geometry{letterpaper,
bindingoffset=0.2in,
left=1in,
right=1in,
top=1in,
bottom=1in,
footskip=.25in}
% -------------------
% Font
% -------------------
%\usepackage[T1]{fontenc}
%\usepackage{charter}

%\usepackage[T1]{fontenc}
%\usepackage{mathpazo}

%\usepackage[bitstream-charter]{mathdesign}
%\usepackage[T1]{fontenc}


% -------------------
% Commands
% -------------------

% Problem Labels
\newcounter{problem}[section]
\newenvironment{problem}[1][\theproblem]{\refstepcounter{problem}\noindent \textbf{Problem~#1.\quad}}{\vskip\smallskipamount}


\newenvironment{solution}{\noindent \textbf{Solution.\quad}}{\vskip\bigskipamount}


% Special Characters
\newcommand{\N}{\mathbb{N}}
\newcommand{\Z}{\mathbb{Z}}
\newcommand{\Q}{\mathbb{Q}}
\newcommand{\R}{\mathbb{R}}
\newcommand{\C}{\mathbb{C}}

\newcommand{\mc}[1]{\mathcal{#1}}

\newcommand{\ob}[1]{\text{ob}(#1)}
\newcommand{\natt}{\Rightarrow}
\newcommand{\iso}{\cong}
\newcommand{\op}{\text{op}}

\newcommand{\Set}{\textbf{Set}}

% Math Operators

% Special Commands



% DOC
\begin{document}
	\assignment{5}

\begin{problem}[46]
Give a direct proof that $H_A \iso H_{A'} \implies A \iso A'$
\end{problem}
\begin{solution}
	So I don't need to put curly braces around all of my subscripts (thanks \LaTeX), I'll replace $A'$ with $B$.  
	We know $H_A : \mc{C}^\op \to \Set$ is a functor, and so isomorphisms are given by natural isomorphisms. Suppose we have such a natural isomorphism $\alpha : H_A \natt H_B$ with components $\alpha_C: H_A(C) \to H_B(C)$ for $C \in \mc{C}^\op$. 
	
	Then, the following diagram commutes for every $f^\op :C \to D$ derived from a $f: D\to C$
	
	% https://q.uiver.app/#q=WzAsNCxbMCwwLCJIX0EoQykgPSBcXG1je0N9KEMsIEEpIl0sWzMsMCwiSF9BKEQpID0gXFxtY3tDfShELCBBKSJdLFszLDIsIkhfQihEKSA9IFxcbWN7Q30oRCwgQikiXSxbMCwyLCJIX0IoQykgPSBcXG1je0N9KEMsIEIpIl0sWzAsMSwiSF9BKGZeXFxvcCkgOj0gLSBcXGNpcmMgZiJdLFsxLDIsIlxcYWxwaGFfRCJdLFswLDMsIlxcYWxwaGFfQyIsMl0sWzMsMiwiSF9CKGZeXFxvcCkgOj0gLSBcXGNpcmMgZiIsMl1d
	\[\begin{tikzcd}
		{H_A(C) = \mc{C}(C, A)} &&& {H_A(D) = \mc{C}(D, A)} \\
		\\
		{H_B(C) = \mc{C}(C, B)} &&& {H_B(D) = \mc{C}(D, B)}
		\arrow["{H_A(f^\op) := - \circ f}", from=1-1, to=1-4]
		\arrow["{\alpha_C}"', from=1-1, to=3-1]
		\arrow["{\alpha_D}", from=1-4, to=3-4]
		\arrow["{H_B(f^\op) := - \circ f}"', from=3-1, to=3-4]
	\end{tikzcd}\]
	
	In particular, for $C=A$, we must have the identity morphism $1_A$. To show that $A\iso B$, we need an invertible morphism $g : A\to B$. 
	Let's chase $1_A$ around this commuting diagram in the hope of getting one...
	% https://q.uiver.app/#q=WzAsOCxbMCwwLCJIX0EoQSkgPSBcXG1je0N9KEEsIEEpIl0sWzMsMCwiSF9BKEQpID0gXFxtY3tDfShELCBBKSJdLFszLDMsIkhfQihEKSA9IFxcbWN7Q30oRCwgQikiXSxbMCwzLCJIX0IoQykgPSBcXG1je0N9KEEsIEIpIl0sWzEsMSwiMV9BIl0sWzIsMSwiZiJdLFsyLDIsIlxcYWxwaGFfQSgxX0EpIFxcY2lyYyBmPVxcYWxwaGFfRChmKSJdLFsxLDIsIlxcYWxwaGFfQSgxX0EpIl0sWzAsMSwiSF9BKGZeXFxvcCkgOj0gLSBcXGNpcmMgZiJdLFsxLDIsIlxcYWxwaGFfRCJdLFswLDMsIlxcYWxwaGFfQSIsMl0sWzMsMiwiSF9CKGZeXFxvcCkgOj0gLSBcXGNpcmMgZiIsMl0sWzQsNSwiIiwwLHsic3R5bGUiOnsidGFpbCI6eyJuYW1lIjoibWFwcyB0byJ9fX1dLFs1LDYsIiIsMCx7InN0eWxlIjp7InRhaWwiOnsibmFtZSI6Im1hcHMgdG8ifX19XSxbNCw3LCIiLDIseyJzdHlsZSI6eyJ0YWlsIjp7Im5hbWUiOiJtYXBzIHRvIn19fV0sWzcsNiwiIiwyLHsic3R5bGUiOnsidGFpbCI6eyJuYW1lIjoibWFwcyB0byJ9fX1dXQ==
	\[\begin{tikzcd}
		{H_A(A) = \mc{C}(A, A)} &&& {H_A(D) = \mc{C}(D, A)} \\
		& {1_A} & f \\
		& {\alpha_A(1_A)} & {\alpha_A(1_A) \circ f=\alpha_D(f)} \\
		{H_B(C) = \mc{C}(A, B)} &&& {H_B(D) = \mc{C}(D, B)}
		\arrow["{H_A(f^\op) := - \circ f}", from=1-1, to=1-4]
		\arrow["{\alpha_A}"', from=1-1, to=4-1]
		\arrow["{\alpha_D}", from=1-4, to=4-4]
		\arrow[maps to, from=2-2, to=2-3]
		\arrow[maps to, from=2-2, to=3-2]
		\arrow[maps to, from=2-3, to=3-3]
		\arrow[maps to, from=3-2, to=3-3]
		\arrow["{H_B(f^\op) := - \circ f}"', from=4-1, to=4-4]
	\end{tikzcd}\]
	
	$\alpha_A$ is a morphism from $H_A(A) \to H_B(A)$, and so $g = \alpha_A(1_A)$ is an element of $\mc{C}(A, B)$. By the invertibility of $\alpha$, we can find $h = \alpha^{-1}_B(1_B)$, a morphism $B\to A$. All that remains is to show that they are inverses. We choose $f = h$, and conclude that the following commutes, which is to say that $gh = 1_B$. Walking backwards along the diagram gives us the statement that $hg = 1_A$. Thus, $A \iso B$. \qed
	% https://q.uiver.app/#q=WzAsOCxbMCwwLCJIX0EoQSkgPSBcXG1je0N9KEEsIEEpIl0sWzMsMCwiSF9BKEIpID0gXFxtY3tDfShCLCBBKSJdLFszLDMsIkhfQihCKSA9IFxcbWN7Q30oQiwgQikiXSxbMCwzLCJIX0IoQykgPSBcXG1je0N9KEEsIEIpIl0sWzEsMSwiMV9BIl0sWzIsMSwiXFxhbHBoYV9CXnstMX0iXSxbMiwyLCJcXGFscGhhX0EoMV9BKSBcXGNpcmMgXFxhbHBoYV9CKDFfQik9XFxhbHBoYV9CKFxcYWxwaGFfQl57LTF9KSA9IDFfQiJdLFsxLDIsIlxcYWxwaGFfQSgxX0EpIl0sWzAsMSwiSF9BKGZeXFxvcCkgOj0gLSBcXGNpcmMgaCJdLFsxLDIsIlxcYWxwaGFfQiJdLFswLDMsIlxcYWxwaGFfQSIsMl0sWzMsMiwiSF9CKGZeXFxvcCkgOj0gLSBcXGNpcmMgaCIsMl0sWzQsNSwiIiwwLHsic3R5bGUiOnsidGFpbCI6eyJuYW1lIjoibWFwcyB0byJ9fX1dLFs1LDYsIiIsMCx7InN0eWxlIjp7InRhaWwiOnsibmFtZSI6Im1hcHMgdG8ifX19XSxbNCw3LCIiLDIseyJzdHlsZSI6eyJ0YWlsIjp7Im5hbWUiOiJtYXBzIHRvIn19fV0sWzcsNiwiIiwyLHsic3R5bGUiOnsidGFpbCI6eyJuYW1lIjoibWFwcyB0byJ9fX1dXQ==
	\[\begin{tikzcd}
		{H_A(A) = \mc{C}(A, A)} &&& {H_A(B) = \mc{C}(B, A)} \\
		& {1_A} & {\alpha_B^{-1}} \\
		& {\alpha_A(1_A)} & {\alpha_A(1_A) \circ \alpha_B(1_B)=\alpha_B(\alpha_B^{-1}) = 1_B} \\
		{H_B(C) = \mc{C}(A, B)} &&& {H_B(B) = \mc{C}(B, B)}
		\arrow["{H_A(f^\op) := - \circ h}", from=1-1, to=1-4]
		\arrow["{\alpha_A}"', from=1-1, to=4-1]
		\arrow["{\alpha_B}", from=1-4, to=4-4]
		\arrow[maps to, from=2-2, to=2-3]
		\arrow[maps to, from=2-2, to=3-2]
		\arrow[maps to, from=2-3, to=3-3]
		\arrow[maps to, from=3-2, to=3-3]
		\arrow["{H_B(f^\op) := - \circ h}"', from=4-1, to=4-4]
	\end{tikzcd}\]
	
	

	
	
%	Suppose $H_A \iso H_{B}$. That is, there exists a morphism $f : H_A \to H_B$ with an inverse $g : H_B \to H_A$. 



\end{solution}




\begin{problem}[49]
	Using the methods in the proof of the Yoneda lemma (NO USING THE ACTUAL STATEMENT OF THE YONEDA LEMMA), prove that the Yoneda embedding is faithful.
\end{problem}
\begin{solution}
	For the Yoneda embedding $H_\bullet: \mc{C} \to [\mc{C}^\op, \Set]$ to be faithful, we need the morphism part of the functor to be injective. That is, $H_f = H_g \implies f = g$. 
	
	Let $f, g: A \to B$ in $\mc{C}(A, B)$. We write $H_f: H_A \natt H_B$, where components $(H_f)_{C}: H_A(C) \to H_A(D)$ for $C$ are given by $f \circ -$, and similarly for $g$. 
	
	As in the proof of the Yoneda lemma, we abuse that natural transformations of representable functors are defined fully by their action on the identity of their representative. 
	
	Suppose that $H_f = H_g$. They must be equal at all components. In particular, for any $C$, $(H_f)_C (h) = f \circ h = (H_g)_C h = g \circ h$. We choose $h = 1_C$. Then, $f = g$.  \qed



\end{solution}




\end{document}
