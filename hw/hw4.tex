\documentclass[12pt]{article}
% -------------------
% Packages
% -------------------
\usepackage{
	amsmath,			% Math Environments
	amssymb,			% Extended Symbols
	amsthm,			    % Theorem Environments
	cancel,			    % Use Cancels
	enumerate,		    % Enumerate Environments
	graphicx,			% Include Images
	lastpage,			% Reference Lastpage
	multicol,			% Use Multi-columns
	multirow,			% Use Multi-rows
	xcolor,			    % Use Colors
	float,
	geometry,
	quiver
	}
\usetikzlibrary{nfold}
% -------------------
% Header & Footer
% -------------------
\usepackage{fancyhdr}

\newcommand{\assignment}[1]{
	\fancypagestyle{title}{
		%Headers
		\fancyhead[L]{Jake Bridge}
		\fancyhead[C]{MATH 418}
		\fancyhead[R]{HW #1}
		\renewcommand{\headrulewidth}{0.2pt}
		%Footers
		\fancyfoot[L]{}
		\fancyfoot[C]{}
		\fancyfoot[R]{}
		\renewcommand{\footrulewidth}{0.0pt}
	}
	\thispagestyle{title}
}
\fancypagestyle{pages}{
	%Headers
	\fancyhead[L]{}
	\fancyhead[C]{}
	\fancyhead[R]{}
	\renewcommand{\headrulewidth}{0.0pt}
	%Footers
	\fancyfoot[L]{}
	\fancyfoot[C]{}
	\fancyfoot[R]{}
	\renewcommand{\footrulewidth}{0.0pt}
}
\headheight=18pt
\footskip=14pt

\pagestyle{pages}

% ---------------
% Page Layout
% ---------------
\geometry{letterpaper,
bindingoffset=0.2in,
left=1in,
right=1in,
top=1in,
bottom=1in,
footskip=.25in}
% -------------------
% Font
% -------------------
%\usepackage[T1]{fontenc}
%\usepackage{charter}

%\usepackage[T1]{fontenc}
%\usepackage{mathpazo}

%\usepackage[bitstream-charter]{mathdesign}
%\usepackage[T1]{fontenc}


% -------------------
% Commands
% -------------------

% Problem Labels
\newcounter{problem}[section]
\newenvironment{problem}[1][\theproblem]{\refstepcounter{problem}\noindent \textbf{Problem~#1.\quad}}{\vskip\smallskipamount}


\newenvironment{solution}{\noindent \textbf{Solution.\quad}}{\vskip\bigskipamount}


% Special Characters
\newcommand{\N}{\mathbb{N}}
\newcommand{\Z}{\mathbb{Z}}
\newcommand{\Q}{\mathbb{Q}}
\newcommand{\R}{\mathbb{R}}
\newcommand{\C}{\mathbb{C}}

\newcommand{\mc}[1]{\mathcal{#1}}

\newcommand{\ob}[1]{\text{ob}(#1)}
\newcommand{\natt}{\Rightarrow}

% Math Operators

% Special Commands



% DOC
\begin{document}
	\assignment{4}
	
	\newcommand{\Set}{\textbf{Set}}

\begin{problem}[35]
	Let $\mc{C}$ be a category, and $X \in \mc{C}$. The slice category $\mc{C} / X$ has objects $(Y, f)$ where $Y \in \mc{C}$ and $f \in \mc{C}(Y, X)$. 
	\begin{enumerate}
		\item Construct the rest of the slice category
		\item For a set $S$ with corresponding discrete category $dS$, $[dS, \Set] \simeq \Set/S$
	\end{enumerate}
\end{problem}


\begin{solution}
	
	The first part of this problem is relatively straightforward. The second is also straightforward, but takes a lot of checking. I use \textbf{Bold} text to break up the work into steps. 
	
	\vspace{1em}
	
	
	\textbf{Motivating Part 1}
	
	The equivalence given in part 2 gives us motivation for the morphisms in the slice category: A Set mod S should look like the functor category between dS and Set, which has morphisms natural transformations between functors from $dS \to \Set$. 
	
	\textbf{What are the morphisms of $\mc{C}/X$?}
	
	
	So, given $(Y, f), (Z, g) \in \mc{C}/X$, we can say that a morphism $F$ must take $Y$ to $Z$ (a morphism in $\mc{C}$) and $f: Y \to X$ to $g: Z \to X$. We draw:
	% https://q.uiver.app/#q=WzAsNCxbMCwwLCJZIl0sWzIsMCwiWCJdLFsyLDIsIlgiXSxbMCwyLCJaIl0sWzAsMSwiZiJdLFsxLDIsIjFfWCJdLFswLDMsIj8iLDJdLFszLDIsImciLDJdXQ==
	\[\begin{tikzcd}
		Y && X \\
		\\
		Z && X
		\arrow["f", from=1-1, to=1-3]
		\arrow["{?}"', from=1-1, to=3-1]
		\arrow["{1_X}", from=1-3, to=3-3]
		\arrow["g"', from=3-1, to=3-3]
	\end{tikzcd}\]
	
	So, our ? ought to be the morphism $h: Y\to Z$ such that $g \circ h = f$. 
	
	That is, $F$ is a morphism $F: Y\to Z$ with $g \circ F = f$. Morphism composition is given by morphism composition in $\mc{C}$, and the identity is the identity in the underlying category as well. 
	
	\textbf{Showing that $\mc{C}/X$ is actually a category}
	
	We must show that $\mc{C} / X$ is a category with this construction. We will first check that morphism composition is sensible, then that identities function how we want. We inherit associativity from the underlying category. 
	
	Given % https://q.uiver.app/#q=WzAsMyxbMCwwLCIoWV8xLCBmXzEpIl0sWzEsMCwiKFlfMiwgZl8yKSJdLFsyLDAsIihZXzMsIGZfMykiXSxbMCwxLCJGIl0sWzEsMiwiRyJdXQ==
	\begin{tikzcd}[cramped]
		{(Y_1, f_1)} & {(Y_2, f_2)} & {(Y_3, f_3)}
		\arrow["F", from=1-1, to=1-2]
		\arrow["G", from=1-2, to=1-3]
	\end{tikzcd}, $G \circ F$ has the right sources and targets (it maps from $Y_1 \to Y_3$). Then, using associativity of morphisms in the underlying category and the condition on morphisms in the slice category, $f_3 \circ (G \circ F) = f_2 \circ F = f_1$. Hence, the composition of two morphisms is a morphism!
	
	Lastly, we check identities. Let \begin{tikzcd}[cramped]
		{(Y_1, f_1)} & {(Y_2, f_2)} 
		\arrow["F", from=1-1, to=1-2]
		\end{tikzcd}. We said that the identity on $(Y_1, f_1)$ ought to be $1_{Y_1}$.  $f_2 \circ (F \circ 1_{Y_1}) = f_2 \circ F = f_1$. This proves right identities. Left identities follow identically. 
	
	We conclude that $\mc{C}/X$ is a category. \qed 
	
	\vspace{1em}
	\textbf{Part 2 Roadmap}
	
	To show equivalence, we can construct two functors, one from $[dS, \Set]$ to $\Set/S$ and one the other way. To define such a functor, we must first show what it does on objects, then morphisms, check that it respects composition and identities, and lastly we will show that the functors compose in a way that is naturally isomorphic to the identity (both ways).  
	
	\textbf{Building the functor} $J: [dS, \Set] \to \Set/s$
	
	
	Given a functor $F: dS \to \Set$, we must send it to a pair $(X, f: X \to S)$ in $\Set/S$. $F$ has two parts: on objects, it maps $s \mapsto Y_s$. On morphisms (of which there are only identities), it maps $1_s \mapsto 1_Y$. We can map $F$ to the pair $(\coprod_S F(s), \pi : \coprod_S F(s) \to S)$ where $\pi(k) = s$ if $k \in F(s)$. This is well defined, because even if $F$ maps two elements $s, s'$ of $dS$ to the same set, we use the disjoint product to identify where each $s, s'$ came from. Let's call the functor performing this operation $J$ for "Jake has no better ideas for notation." 
	
We must see what $J$ does on morphisms. Morphisms in the functor category are natural transformations. Given $\alpha: F \natt G$, we define $J(\alpha) = \coprod_S \alpha_s$ where $\alpha_s$ are morphisms $p_s: F(s) \to G(s)$ in $\Set$. This has the proper source and target. $dS$ is discrete, so the only $f: s \to s'$ is the identity on $s$. The naturality square of $\alpha$ just says that % https://q.uiver.app/#q=WzAsNCxbMCwwLCJGcyJdLFsyLDAsIkZzIl0sWzIsMiwiR3MiXSxbMCwyLCJHcyJdLFswLDEsIjFfe0ZzfSJdLFsxLDIsIlxcYWxwaGFfe3N9Il0sWzAsMywiXFxhbHBoYV9zIiwyXSxbMywyLCIxX3tHc30iLDJdLFswLDIsIlxcYWxwaGFfcyIsMV1d
\[\begin{tikzcd}
	Fs && Fs \\
	\\
	Gs && Gs
	\arrow["{1_{Fs}}", from=1-1, to=1-3]
	\arrow["{\alpha_s}"', from=1-1, to=3-1]
	\arrow["{\alpha_s}"{description}, from=1-1, to=3-3]
	\arrow["{\alpha_{s}}", from=1-3, to=3-3]
	\arrow["{1_{Gs}}"', from=3-1, to=3-3]
\end{tikzcd}\]

This obviously commutes. 

 Let $JF_\pi$ be the projection part of $JF$, and similarly for $JG$. We need to first show that $JG_\pi \circ J\alpha = JF_\pi$. Suppose $k \in F(s_0)$. Then, $p_s(k) \in G(s_0)$
 
 Then $JF_\pi(k) = s_0$, and $JG_\pi \circ \coprod_S p_s(k) = s_0$. 
 
 
 \textbf{Showing $J$ is a functor}
 
 
	Now, we show $J$ is a functor: The identity natural transformation $1_F$ has components identities ${1_F}_s = 1_{Fs}$. We need that $J(1_F) = 1_{JF}$. So, we write $J(1_F) = \coprod_S 1_{F(s)} = 1_{\coprod_S F(s)} = 1_{JF}$
	
	Now, consider $\alpha, \beta: F\natt G$ where $F, G \in [dS, \Set]$ and their vertical composition $\alpha \circ \beta$. $J(\alpha \circ \beta) = \coprod_S \alpha_s \circ \beta_s = \coprod_S \alpha_s \circ \coprod_S \beta_s = J(\alpha) \circ J(\beta)$.
	
	Thus, $J$ is a functor.
	
\textbf{Building the functor $K: \Set/S \to [dS, \Set]$}

Given $(Y, f: Y \to S)$ in the slice category, we must map it to a functor $F$ from $dS \to \Set$. 
First, we build the functor $F$, then show that it is a functor. We define $F$ on objects to take $s \mapsto f^{-1}(s)$. On morphisms, the only morphisms of $dS$ are the identities, and so they go to the corresponding identities. The only morphism composition that $F$ needs to respect is repeated identity composition, which it clearly does. Hence, $F$ is a functor. That is, $K(Y, f) = F$. 

Now, $K$ must take morphisms of $\Set/S$ to natural transformations between functors between $dS\to \Set$. Luckily, we had defined our morphisms of the slice category so that this would make sense! Given a morphism $H : (Y, f) \to (Z, g)$, where $K(Y, f) = F$ and $K(Z, g) = G$, we think of $H$ as a morphism $h: Y\to Z$ with $g \circ h = f$. 

So, $K(H)$ must be a natural transformation given by components $KH_s : F(s) \to G(s)$. By the same argument with $J$, the naturality square automatically commutes as there is only the identity morphism from each $s\to s$ in $dS$. We can let $KH_s = h|_{f^{-1}(s)}$. We know that $g\circ h = f$, so if $x \in f^{-1}(s)$, then $g(h(x)) = s$, so $h(x) \in g^{-1}$. Hence $KH_s$ has the proper source ($f^{-1}(s)$) and target ($g^{-1}(s)$). 

\textbf{Showing $K$ is a functor}
Identities: $K(1_X)_s = 1_X|_{f^{-1}(s)} = 1_{KX}$. 

Composition: $K(X \xrightarrow{F} Y \xrightarrow{G} Z)_S = (g \circ f)_{(g \circ f)^{-1}(s)}$.
$x \in (f\circ g)^{-1}(s) \iff fx \in g^{-1}(s)$. So, we can split the restriction to $g|_{g^{-1}(s)} \circ f|_{f^{-1}(s)}$. This is $KG \circ KF$. 

\textbf{Proving equivalence}
We need to show that $KJ$ and $JK$ are both naturally isomorphic to the identity, beginning with JK. 
\begin{equation*}
	\begin{split}
JK((Y, f: Y \to S) \xrightarrow{F} (Z, g: Z\to S)) \\
= J(f^{-1}(S) \xrightarrow{F|_{f^{-1}(S)}} g^{-1}(S)) \\
= (\coprod_S f^{-1}(s), \pi_f) \xrightarrow{\coprod_S F|_{f^{-1}(s)}} (\coprod_S g^{-1}(s), \pi_g)
	\end{split}
\end{equation*}

We must find an isomorphism $\Phi$ from $\Set / S \natt \Set / S$ taking $(Y, f)$ to $\coprod_S f^{-1}(s)$ such that the following diagram commutes. % % https://q.uiver.app/#q=WzAsNCxbMCwzLCIoWSwgZikiXSxbNCwzLCIoWiwgZykiXSxbMCwwLCIoXFxjb3Byb2RfUyBmXnstMX0ocyksIFxccGlfZikiXSxbNCwwLCIoXFxjb3Byb2RfUyBnXnstMX0ocyksIFxccGlfZykiXSxbMCwxLCJoIiwyXSxbMiwwLCJcXFBoaV97WSwgZn0iLDJdLFszLDEsIlxcUGhpX3taLCBnfSJdLFsyLDMsIlxcY29wcm9kX1MgaHxfe2Zeey0xfShzKX0iXV0=
\[\begin{tikzcd}
	{(\coprod_S f^{-1}(s), \pi_f)} &&&& {(\coprod_S g^{-1}(s), \pi_g)} \\
	\\
	\\
	{(Y, f)} &&&& {(Z, g)}
	\arrow["{\coprod_S h|_{f^{-1}(s)}}", from=1-1, to=1-5]
	\arrow["{\Phi_{Y, f}}"', from=1-1, to=4-1]
	\arrow["{\Phi_{Z, g}}", from=1-5, to=4-5]
	\arrow["h"', from=4-1, to=4-5]
\end{tikzcd}\]

Given some $(s, y) \in \coprod_S f^{-1}(s)$ (thinking about the disjoint union as a tagged union), we take $\Phi_{Y, f}((s, y), \pi_f) = (y, f)$.

 Then, $(h\circ \Phi_{Y, f})((s, y), \pi_f) = (h(y), g)$ and $( \Phi_{Z, g} \circ \coprod_S h|_{f^{-1}(s)})((s, y), \pi_f) = \Phi_{Z, g}((h(s), h(y)) , \pi_g) = (h(y), g)$, so the square commutes. 
 
 Moreover, $\Phi$ is invertible: we can define $\Phi^{-1}_{Y, f}(y, f) = ((s, y), \pi_f)$ where $s$ is an element (choose one!) such that $y \in f^{-1}(s)$. Then, $\Phi \circ \Phi^{-1}$ is clearly the identity, and $\Phi^{-1} \circ \Phi$ is isomorphic to the identity (by choosing the "right" $s$). We can just redefine $\Phi$ to choose such $s$, and this gives us that the natural transformation $\Phi$ is in fact a natural isomorphism from $JK$ to the identity. Hence, $JK \cong 1$. 
 
 Lastly, we show that $KJ \cong 1$. 
 
 \begin{equation*}
 	\begin{split}
 		KJ(F \natt_\alpha G) = \\
 		K(   (\coprod_S F(s), \pi_F) \xrightarrow{\coprod_S \alpha_s} (\coprod_S G(s), \pi_G)  ) = \\
 		\pi_F^{-1}F(s)      \natt_\alpha          \pi_G^{-1}G(s) = \\
 	\end{split}
 \end{equation*}

Given (the object part of a) functor: $f : dS \to Y$, we follow it around: $KJ(f) =  K(\coprod_S f(s), \pi_f) = \pi_f^{-1}$. $KJ(f)(s) = \{(s, y) | f(s) = y)\}$. We can call this set $Y_s$ where $Y \cong \coprod_S Y_s$ and the isomorphism is natural in $\Set$. If we show this isomorphism, we will have that $KJ$ takes $F$ to something isomorphic to $F$. I will not go through the details of that-- effectively, the tagged inverse images partition $Y$ and we just take the functor that glues the pieces together. We glue the morphisms together in the same way. That is, that $KJ$ is equivalent to the identity on elements. By the same logic, $KJ$ is equivalent the identity on morphisms (natural transformations). 

Hence, $KJ \cong 1$, so $[dS, \Set] \simeq \Set/S$, and we are done! \qed


	
\end{solution}




\begin{problem}[38]
	Let $\mathcal{K}$ be a 2-category with horizontal composition given by a functor 
	\[\mathcal{K}(B, C) \times \mathcal{K}(A, B) \to \mathcal{K}(A, C)\]
	
	Prove that the composition functor preserving vertical composition of 2-cells is the same equality as in the interchange law. 
\end{problem}
\begin{solution}
	Vertical composition of 2-cells is composition of morphisms in the category of 1 and 2 cells (composing 2-cells along 1-cells). Horizontal composition of 2-cells is like composing 2-cells along 0-cells. 
	
	For horizontal composition to be a functor, it must respect identities and composition. In particular, respecting composition means that if $f, f', f''$ are objects (1-cells) in $\mathcal{K}(A, B)$ with morphisms (2-cells) $\alpha: f\natt f'$ and $\alpha': f' \natt f''$ and similarly in $\mathcal{K}(B, C)$ has morphisms $\beta: g \natt g'$ and $\beta' : g' \natt g''$. 
	
	The functor $*$ takes objects of the product category (pairs of 1-cells) $(g, f)$ to the object (1-cell) $g \circ f$ in the category $\mathcal{K}(A, C)$. 
	
	On morphisms, $*$ takes morphisms in the product category (pairs of 2-cells) $(\beta, \alpha)$ to the 2-cell $\beta * \alpha$ in $\mathcal{K}(A, C)$.
	
	For $*$ to respect vertical composition (which is to say, composition in the product and target categories), first notice that $(\beta', \alpha') \circ (\beta, \alpha) = (\beta' \circ \beta, \alpha' \circ \alpha)$. The functor $*(a, b) = a * b$ must have that $*((\beta', \alpha') \circ (\beta, \alpha)) = *((\beta', \alpha')) \circ *((\beta, \alpha))$. This is precisely the interchange law. \qed 
	
%	We observe the abomination:
%	% https://q.uiver.app/#q=WzAsMyxbMCwwLCJBIl0sWzMsMCwiQiJdLFs2LDAsIkMiXSxbMCwxLCJmJyIsMV0sWzAsMSwiZicnIiwxLHsiY3VydmUiOjR9XSxbMCwxLCJmIiwxLHsiY3VydmUiOi00fV0sWzEsMiwiZyciLDFdLFsxLDIsImciLDEseyJjdXJ2ZSI6LTR9XSxbMSwyLCJnJyciLDEseyJjdXJ2ZSI6NH1dLFszLDQsIlxcYWxwaGEnIiwyLHsic2hvcnRlbiI6eyJzb3VyY2UiOjIwLCJ0YXJnZXQiOjIwfX1dLFs1LDMsIlxcYWxwaGEiLDIseyJzaG9ydGVuIjp7InNvdXJjZSI6MjAsInRhcmdldCI6MjB9fV0sWzcsNiwiXFxiZXRhIiwwLHsic2hvcnRlbiI6eyJzb3VyY2UiOjIwLCJ0YXJnZXQiOjIwfX1dLFs2LDgsIlxcYmV0YSciLDAseyJzaG9ydGVuIjp7InNvdXJjZSI6MjAsInRhcmdldCI6MjB9fV0sWzUsNCwiXFxhbHBoYScgXFxjaXJjIFxcYWxwaGEiLDAseyJsYWJlbF9wb3NpdGlvbiI6NDAsImN1cnZlIjotMiwic2hvcnRlbiI6eyJzb3VyY2UiOjEwLCJ0YXJnZXQiOjEwfX1dLFs3LDgsIlxcYmV0YScgXFxjaXJjIFxcYmV0YSIsMix7ImxhYmVsX3Bvc2l0aW9uIjo0MCwiY3VydmUiOjIsInNob3J0ZW4iOnsic291cmNlIjoxMCwidGFyZ2V0IjoxMH19XSxbOSwxMiwiXFxiZXRhJyAqIFxcYWxwaGEnIiwyLHsiY3VydmUiOjN9XSxbMTAsMTEsIlxcYmV0YSAqXFxhbHBoYSIsMCx7ImN1cnZlIjotM31dLFsxMywxNCwiKFxcYmV0YScgXFxjaXJjIFxcYmV0YSkqKFxcYWxwaGEnIFxcY2lyYyBcXGFscGhhKSIsMix7Im9mZnNldCI6MywiY3VydmUiOjF9XV0=
%	\[\begin{tikzcd}[row sep=6em, column sep=large]
%		A &&& B &&& C
%		\arrow[""{name=0, anchor=center, inner sep=0}, "{f'}"{description}, from=1-1, to=1-4]
%		\arrow[""{name=1, anchor=center, inner sep=0}, "{f''}"{description}, curve={height=24pt}, from=1-1, to=1-4]
%		\arrow[""{name=2, anchor=center, inner sep=0}, "f"{description}, curve={height=-24pt}, from=1-1, to=1-4]
%		\arrow[""{name=3, anchor=center, inner sep=0}, "{g'}"{description}, from=1-4, to=1-7]
%		\arrow[""{name=4, anchor=center, inner sep=0}, "g"{description}, curve={height=-24pt}, from=1-4, to=1-7]
%		\arrow[""{name=5, anchor=center, inner sep=0}, "{g''}"{description}, curve={height=24pt}, from=1-4, to=1-7]
%		\arrow[""{name=6, anchor=center, inner sep=0}, "{\alpha'}"', between={0.2}{0.8}, Rightarrow, from=0, to=1]
%		\arrow[""{name=7, anchor=center, inner sep=0}, "\alpha"', between={0.2}{0.8}, Rightarrow, from=2, to=0]
%		\arrow[""{name=8, anchor=center, inner sep=0}, "{\alpha' \circ \alpha}"{pos=0.47}, curve={height=-12pt}, between={0.1}{0.9}, Rightarrow, from=2, to=1]
%		\arrow[""{name=9, anchor=center, inner sep=0}, "\beta", between={0.2}{0.8}, Rightarrow, from=4, to=3]
%		\arrow[""{name=10, anchor=center, inner sep=0}, "{\beta'}", between={0.2}{0.8}, Rightarrow, from=3, to=5]
%		\arrow[""{name=11, anchor=center, inner sep=0}, "{\beta' \circ \beta}"'{pos=0.47}, curve={height=12pt}, between={0.1}{0.9}, Rightarrow, from=4, to=5]
%		\arrow["{\beta' * \alpha'}"', curve={height=36pt}, Rightarrow, scaling nfold=3, from=6, to=10]
%		\arrow["{\beta *\alpha}", curve={height=-18pt}, Rightarrow, scaling nfold=3, from=7, to=9]
%		\arrow["{(\beta' \circ \beta)*(\alpha' \circ \alpha)}"', shift right=3, curve={height=6pt}, Rightarrow, scaling nfold=3, from=8, to=11]
%	\end{tikzcd}\]
%	
%	That is, the path from $\alpha$ to $\beta$ vertically composed with the path from $\alpha'$ to $\beta'$ should be the same as the path from $\alpha' \circ \alpha$ to $\beta' \circ \beta$. 
	

	

\end{solution}


\begin{problem}[41]
Write out the details for the 2-category of posets, order-preserving functions, and function domination
\end{problem}
\begin{solution}
	
	The solution is effectively recognizing that order preserving functions are themselves a poset under function domination, and then using uniqueness of 2-cells to verify that all of the 2-category axioms hold. 
	
	\newcommand{\Poset}{\textbf{Poset}}
	The 0-cells of $\Poset$ are posets. Between any two posets $(A, \leq_A)$ and $(B, \leq_B)$, there is a category $\Poset(A, B)$ (I drop the $\leq_A$ and $\leq_B$---they should be clear from here on out.)
	
	A 1-cell in $\Poset(A, B)$ is an order preserving function $f: A \to B$ where $a \leq_A a' \implies f(a) \leq_B f(a')$. A 2-cell between $f : A \to B$ and $g: A \to B$ is a unique morphism $\heartsuit_{f, g}$ if $f(a) \leq_B g(a) \forall a \in A$. Composition is given by $\heartsuit_{g, h} \circ \heartsuit_{f, g} = \heartsuit_{f, h}$ and is well-defined by the transitive property of inequalities (if $f(a) \leq_B g(a)$ and $g(a) \leq_B h(a)$ for all $a \in A$, then $f(a) \leq_B h(a)$ and so $\heartsuit_{f, h}$ exists)
	
	First, we show that $\Poset(A, B)$ is a category. The identity morphism $\heartsuit_{f, f}$ exists and is clearly an identity. Associativity is also obvious. Hence, $\Poset(A, B)$ is a category. 
	
	
	Next, we will show that there is a functor from the one-object category to the identity 1 and 2-cells of an 0-cell $A$. This is effectively the question if there exists a 1-cell that serves as an identity for each 0-cell. The identity function $1_A$ on $A$ is the relevant 1-cell. Then, the 2-cell in question is $\heartsuit_{1_A, 1_A}$. 
	
	Now, we need to define the horizontal composition functor $*$. I'll write $f * g$ for $*(f, g)$. Given posets $A, B, C$ and order preserving functions (1-cells) $f \in \Poset(A, B), g \in \Poset(B, C)$, we define $g * f = g \circ f$. Moreover, given a 2-cell $\heartsuit_{f, f'} \in \Poset(A, B)(f, f')$ and similarly for $g, g'$, we define $\heartsuit_{g, g'} * \heartsuit_{f, f'} = \heartsuit_{g \circ f, g' \circ f'}$. This is well defined by uniqueness of 2-cells and exists by transitivity of the $\leq$ operation.
	
	Horizontal composition is associative on objects by associativity of 1-morphisms, and satisfies the identity axioms similarly. 
	
	 As there's really only one choice to make at each step of composition, and $\leq$ is a transitive operation, horizontal composition is associative on morphisms. Moreover, horizontally composing identities yields an identity, because the identity on $g * f$ is the unique morphism $\heartsuit_{g \circ f, g \circ f}$, which is $\heartsuit_{g, g} * \heartsuit_{f, f}$. The same logic tells us that $*$ is indeed a functor:
	 
	 Let $\beta = \heartsuit_{g, g'}, \beta' = \heartsuit_{g', g''}$ in the category $\Poset(B, C)$ and similarly for $\alpha$. Then
	 
	 $(\beta' \circ \beta) * (\alpha' \circ \alpha) = \heartsuit_{g, g''} * \heartsuit_{f, f''} = \heartsuit_{g \circ f, g'' \circ f''} = \heartsuit_{g' \circ f', g'' \circ f''} \circ \heartsuit_{g \circ f, f' \circ g'} = (\beta' * \alpha') \circ (\beta * \alpha)$
	 
	 So, horizontal composition is a functor (interchange law). 
	 
	 So, the 2-category $\Poset$ as defined is indeed a 2-category. 
	 Ta-da! \qed
%	The identity is the identity function, and morphism composition is given by function composition. Associativity follows from function associativity, and identity laws from the definition of an identity function. 
%	viewed as categories (a unique morphism $f_{A, B}: A \to B$ between any elements where $A \leq B$). 
	
	
\end{solution}

%
%\begin{problem}[44]
%	Let $F: \mathcal{A} \to \mathcal{B}$ be a functor and $G_1, G_2 : \mathcal{B} \to \mathcal{C}$ a pair of functors with a natural transformation $\alpha: G_1 \natt G_2$. We will write $\alpha F = \alpha * 1_F$ and call it "the whiskering of $\alpha$ by $F$. 
%	
%	\begin{enumerate}
%		\item Prove that for transformations $\beta, \alpha$ for which $\beta * \alpha$ is defined, this composite can be derived from vertical composition and whiskering (on both sides).
%		
%		\item Rewrite the definition of a 2-category using whiskering instead of a composition functor. 
%	\end{enumerate}
%\end{problem}
%\begin{solution}
%	
%\end{solution}


\end{document}
